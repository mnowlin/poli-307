\hypertarget{covid-19}{%
\section{COVID-19}\label{covid-19}}

The COVID-19 pandemic is still ongoing. \textbf{The College of
Charleston requires that masks be worn while indoors and you must wear a
mask at all times while in class.} Although vaccinations are currently
not required, \emph{I ask you to be respectful of the health and safety
of others}. If you have not received the \textbf{COVID-19 vaccine
(including a booster), which is safe, free, and effective, please
consider doing so immediately}. Information about the vaccine is
available from the
\href{https://scdhec.gov/covid19/covid-19-vaccine}{SCDHEC website} and
information about where and when to obtain a vaccine is also available
on the SCDHEC website \href{https://vaxlocator.dhec.sc.gov/}{vaccine
locator page}.

\hypertarget{course-description}{%
\section{Course Description}\label{course-description}}

From the CofC catalog:

\begin{quote}
This course is intended to familiarize students with various ethical
frameworks, analytical tools and policy instruments that can be used to
evaluate environmental problems and policy options. Specific issues may
include citizen participation, environmental equity, the uses and abuses
of cost/benefit analysis, science and uncertainty in environmental
policy development and the use of regulatory requirements vs.~market
mechanisms for environmental protection.
\end{quote}

\vspace{0.1in}

\noindent This course is an advanced undergraduate course. We will
examine the central dimensions of environmental politics and policy in
the United States.

\vspace{0.1in}

\noindent The course will provide an overview of the development of
environmental policy issues and environmental politics in the US. The
first part of the course will provide an introduction to environmental
policy and politics in the US and will provide a theoretical base for
understanding environmental policy change and development; examine the
fundamental beliefs and attitudes that have shaped environmental
policies; and the major political institutions in the U.S. that
conceive, design, implement, and revise environmental policies. The
second part of the course will examine environmental policy design
including regulation and market-based approaches. Finally, we will
examine several environmental issues including pollution, land
management, and energy.

\vspace{0.1in}

\noindent Laptops are allowed, but discouraged. Phones are only to be
used to answer quiz questions. \emph{I encourage you to take notes by
hand, with pen and paper}.
\href{https://www.nytimes.com/2017/11/27/learning/should-teachers-and-professors-ban-student-use-of-laptops-in-class.html}{You
learn better that way}. I recommend taking notes using the
\href{http://www.usu.edu/arc/idea_sheets/pdf/note_taking_cornell.pdf}{Cornell
Method}. Also, lecture slides will generally \textbf{not} be made
available outside of class.

\hypertarget{course-prerequisites}{%
\subsection{Course Prerequisites}\label{course-prerequisites}}

There are no prerequisites for this course.

\hypertarget{attendance-policy}{%
\subsection{Attendance Policy}\label{attendance-policy}}

Attendance will not be taken; however, a lack of attendance will result
in missed quiz questions. Additionally, lecture slides will \emph{not}
be made available outside of class. \textbf{Do not come to class if you
feel ill or if you have been exposed to COVID-19, regardless of how you
feel. I am happy to meet with you to discuss material you missed.}

\hypertarget{course-goals-and-learning-objectives}{%
\section{Course Goals and Learning
Objectives}\label{course-goals-and-learning-objectives}}

The goals for this course are to:

\begin{itemize}
\item
  Develop an understanding of the evolution of environmental policy and
  politics in the U.S.
\item
  Develop an understanding of the major policymaking institutions
  including the Congress, the President, Executive Agencies, the Courts,
  and their role in environmental policymaking.
\item
  Develop an understanding of the process of policymaking in the U.S.
  with regard to environmental issues.
\item
  Develop in-depth knowledge about several topics within the broad field
  of environmental policy.
\end{itemize}

\hypertarget{required-materials}{%
\section{Required Materials}\label{required-materials}}

The following materials are \textbf{required}.

\hypertarget{book}{%
\subsection{Book}\label{book}}

We will be using the following book, which will be available on
\href{https://lms.cofc.edu}{OAKS} for no charge. \emph{You are not
expected or required to purchase the book}. Chapters will be available
on \href{https://lms.cofc.edu}{OAKS} under \emph{readings} for each week
as needed.

\begin{itemize}
\item
  Rinfret, Sara R., and Michelle C. Pautz. 2019. \emph{US Environmental
  Policy in Action}. 2nd ed.~New York, NY: Palgrave Macmillan.
\item
  Note that additional required readings will be available on
  \href{https://lms.cofc.edu}{OAKS}
\end{itemize}

\hypertarget{news}{%
\subsection{News}\label{news}}

\begin{itemize}

\item
  \faNewspaperO \hspace{0.005in} \emph{The Environment in the News}: We
  will discuss current news events related to environmental policy
  issues in class. In addition, you will be required to choose an
  environmental problem to examine throughout the semester. To keep
  current and find a topic, I suggest you check these sites frequently
  and/or subscribe to the email lists:

  \begin{itemize}
  
  \item
    \href{https://www.washingtonpost.com/politics/powerpost/the-energy-202/}{The
    Energy 202 (Washington Post)}
  \item
    \href{https://www.politico.com/morningenergy/}{Morning Energy
    (POLITICO)}
  \item
    \href{https://www.nytimes.com/column/climate-fwd}{Climate Fwd (New
    York Times)}
  \end{itemize}
\end{itemize}

\hypertarget{poll-everywhere}{%
\subsection{Poll Everywhere}\label{poll-everywhere}}

You are required to set-up an account and register your phone with Poll
Everywhere. Instructions for setting up your account with Poll
Everywhere will be on \href{https://lms.cofc.edu}{OAKS}

\begin{itemize}
\item
  \emph{There is no cost to use Poll Everywhere for this class}
\item
  I encourage you to review the materials
  \href{https://blog.polleverywhere.com/students-poll-everywhere-101/}{here}
\end{itemize}

\hypertarget{oaks}{%
\subsection{OAKS}\label{oaks}}

OAKS, including Gradebook, will be used for this course throughout the
semester to provide the syllabus and class materials and grades for each
assignment, which will be regularly posted.

\hypertarget{course-requirements-and-grading}{%
\section{Course Requirements and
Grading}\label{course-requirements-and-grading}}

Performance in this course will be evaluated on the basis of in-class
quiz questions, a mid-term exam, a final exam, and a policy brief.
Points will be distributed as follows:

\vspace{0.1in}
\begin{tabular}{ l l}
\hline
Assignment & Possible Points \\ 
\hline
Quiz Questions & 200 points total \\
Mid-Term Exam & 100 points \\ 
Final Exam & 100 points \\
Policy Brief &  350 points total \\
\hline
Total &  750 points \\
\hline
\end{tabular}

\hypertarget{assignments}{%
\subsection{Assignments}\label{assignments}}

All due dates for assignments are on the following \emph{Course
Schedule} and will also be posted on \href{https://lms.cofc.edu}{OAKS}.
Both the mid-term and final exams will be taken online; however, we will
NOT be using any exam proctoring service. The exams will be taken on
OAKS just like the weekly content quizzes.

\vspace{0.1in}

\noindent \emph{Quiz Questions}: There will be 1 to 2 quiz questions
given during each class period and the questions will be answered using
Poll Everywhere on your phone. \emph{You must be present in class to be
able to answer the questions}. These questions will cover material from
the readings and/or class discussion. Each question will be worth 5
points and can not be made up if you miss class. However, \textbf{I will
add up to 25 points to your quiz questions grade at the end of the
course.}

\vspace{0.1in}

\noindent \emph{Mid-term}: The mid-term exam will be given on
\textbf{Thursday March 3} and will be \textbf{taken on OAKS}. All
material from the readings, lectures, and in-class discussions are fair
game for the mid-term exam. The exam will be multiple choice, short
answer, and short essay.

\vspace{0.1in}

\noindent \emph{Final Exam}: \textbf{The final exam period is Thursday
April 28 from 1:00pm to 3:00pm} and it will be \textbf{taken on OAKS}.
The final will NOT be comprehensive and all material from the readings,
lectures, and in-class discussions \emph{since the mid-term} are fair
game. The exam will be multiple choice, short answer, and short essay.

\hypertarget{environmental-policy-brief}{%
\subsubsection{Environmental Policy
Brief}\label{environmental-policy-brief}}

You will pick an environmental problem on which to focus for a policy
brief. A policy brief provides an overview of an environmental problem
as well as possible policies to address that problem. The policy brief
consists of the following four assignments. Further instructions for
each assignment is provided on \href{https://lms.cofc.edu}{OAKS} under
Content -\textgreater{} Assignments -\textgreater{} Environmental Policy
Brief.

\begin{itemize}
\item
  \faInbox \hspace{0.005in} \textbf{Topic Selection},
  \faCalendar \hspace{0.005in} \textbf{February 3}, \emph{25 points}.
\item
  \faInbox \hspace{0.005in} \textbf{Annotated Bibliography},
  \faCalendar \hspace{0.005in} \textbf{February 24}, \emph{100 points}.
\item
  \faInbox \hspace{0.005in} \textbf{Fact Sheet},
  \faCalendar \hspace{0.005in} \textbf{March 31}, \emph{100 points}.
\item
  \faInbox \hspace{0.005in} \textbf{Policy Brief},
  \faCalendar \hspace{0.005in} \textbf{April 21}, \emph{125 points}.
\end{itemize}

\hypertarget{late-work-policy}{%
\paragraph{Late Work Policy}\label{late-work-policy}}

Late work is subject to a 48-hour grace period, and after that will be
penalized 10\% each day (24 hr period) it is late, up to 3 days. After 3
days the assignment will not be accepted. For example, if an assignment
is due Thursday at 2:00 PM, the grace period ends on Saturday at 2:00 PM
and it is late as of 2:01 PM and you lose 10\%. After Sunday at 2:01 PM
you lose another 10\%, after Monday at 2:01 PM another 10\%, and no work
will be accepted after Tuesday at 2:00 PM. \emph{No late work will
accepted 72 hrs after the assignment due date and time}.

\hypertarget{grading-scale}{%
\subsection{Grading Scale}\label{grading-scale}}

There are \textbf{750} possible points for this course. Grades will be
allocated based on your earned points and calculated as a percentage of
\textbf{750}. A: 94 to 100\%; A-: 90 to 93\%; B+: 87 to 89\%; B: 83 to
86\%; B-: 80 to 82\%; C+: 77 to 79\%; C: 73 to 76\%; C-: 70 to 72\%; D+:
67 to 69\%; D: 63 to 67\%; D-: 60 to 62\%; F: 59\% and below
